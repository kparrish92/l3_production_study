% Options for packages loaded elsewhere
\PassOptionsToPackage{unicode}{hyperref}
\PassOptionsToPackage{hyphens}{url}
%
\documentclass[
  12pt,
]{article}
\usepackage{amsmath,amssymb}
\usepackage{lmodern}
\usepackage{ifxetex,ifluatex}
\ifnum 0\ifxetex 1\fi\ifluatex 1\fi=0 % if pdftex
  \usepackage[T1]{fontenc}
  \usepackage[utf8]{inputenc}
  \usepackage{textcomp} % provide euro and other symbols
\else % if luatex or xetex
  \usepackage{unicode-math}
  \defaultfontfeatures{Scale=MatchLowercase}
  \defaultfontfeatures[\rmfamily]{Ligatures=TeX,Scale=1}
\fi
% Use upquote if available, for straight quotes in verbatim environments
\IfFileExists{upquote.sty}{\usepackage{upquote}}{}
\IfFileExists{microtype.sty}{% use microtype if available
  \usepackage[]{microtype}
  \UseMicrotypeSet[protrusion]{basicmath} % disable protrusion for tt fonts
}{}
\makeatletter
\@ifundefined{KOMAClassName}{% if non-KOMA class
  \IfFileExists{parskip.sty}{%
    \usepackage{parskip}
  }{% else
    \setlength{\parindent}{0pt}
    \setlength{\parskip}{6pt plus 2pt minus 1pt}}
}{% if KOMA class
  \KOMAoptions{parskip=half}}
\makeatother
\usepackage{xcolor}
\IfFileExists{xurl.sty}{\usepackage{xurl}}{} % add URL line breaks if available
\IfFileExists{bookmark.sty}{\usepackage{bookmark}}{\usepackage{hyperref}}
\hypersetup{
  pdftitle={L3 French Voice-Onset Time at first exposure by Spanish-English bilinguals.},
  pdfauthor={Kyle Parrish},
  hidelinks,
  pdfcreator={LaTeX via pandoc}}
\urlstyle{same} % disable monospaced font for URLs
\usepackage[margin=1in]{geometry}
\usepackage{longtable,booktabs,array}
\usepackage{calc} % for calculating minipage widths
% Correct order of tables after \paragraph or \subparagraph
\usepackage{etoolbox}
\makeatletter
\patchcmd\longtable{\par}{\if@noskipsec\mbox{}\fi\par}{}{}
\makeatother
% Allow footnotes in longtable head/foot
\IfFileExists{footnotehyper.sty}{\usepackage{footnotehyper}}{\usepackage{footnote}}
\makesavenoteenv{longtable}
\usepackage{graphicx}
\makeatletter
\def\maxwidth{\ifdim\Gin@nat@width>\linewidth\linewidth\else\Gin@nat@width\fi}
\def\maxheight{\ifdim\Gin@nat@height>\textheight\textheight\else\Gin@nat@height\fi}
\makeatother
% Scale images if necessary, so that they will not overflow the page
% margins by default, and it is still possible to overwrite the defaults
% using explicit options in \includegraphics[width, height, ...]{}
\setkeys{Gin}{width=\maxwidth,height=\maxheight,keepaspectratio}
% Set default figure placement to htbp
\makeatletter
\def\fps@figure{htbp}
\makeatother
\setlength{\emergencystretch}{3em} % prevent overfull lines
\providecommand{\tightlist}{%
  \setlength{\itemsep}{0pt}\setlength{\parskip}{0pt}}
\setcounter{secnumdepth}{-\maxdimen} % remove section numbering
\usepackage{tipa}
\usepackage{xcolor}
\ifluatex
  \usepackage{selnolig}  % disable illegal ligatures
\fi

\title{L3 French Voice-Onset Time at first exposure by Spanish-English
bilinguals.}
\author{Kyle Parrish}
\date{10/9/21}

\begin{document}
\maketitle

\textbf{Name:} Kyle Parrish

\textbf{Affiliation:} Rutgers University

\textbf{Email:}
\href{mailto:kyle.parrish@rutgers.edu}{\nolinkurl{kyle.parrish@rutgers.edu}}

\textbf{Paper Title:} The VOT productions of L3 French by
Spanish-English bilinguals at first exposure.

\newpage

\hypertarget{abstract}{%
\section{Abstract}\label{abstract}}

The present study investigates bilinguals' first exposure to an L3 that
they do not yet know in order to inform debates in L3 phonological
acquisition. Specifically, participants who speak L1 Mexican Spanish and
L2 English were given French words to repeat in order to examine whether
the phonology of their first language or their second language plays a
greater role in the starting point of L3 phonological acquisition.
Previous studies have shown that, in the case of voice-onset time (VOT),
L3 learners are primarily influenced by the L2 (Wrembel, 2010; Llama et
al., 2010), or by hybrid values between the L1 and the L2 (Wrembel,
2011; Llama \& Cardoso, 2018). One potential reason for the variation in
findings may stem from small sample sizes in L3 research. Small sample
size is associated with an increased risk of committing type 1 or type 2
errors (Brysbaert, 2020), and it is possible that the variation in the
effects found in the literature to date could be explained by sampling
error, in addition to varied methodology. To address these potential
issues, the present study employs a higher sample size (n = 75) of
absolute beginners and uses tests of equivalence (Lakens, 2017) in order
to determine whether a group trend of the use L1 and/or L2 phonology
exists in the pronunciation of L3 words at the first exposure.

A total of 91 participants completed an L3 shadowing task in which they
repeated 8 voiceless stop-initial words in all three places of
articulation, and elicited production tasks in Spanish and English of 9
words each. All stimuli were either one or two syllable words and were
stressed on the first syllable. Words were presented in isolation and in
a random order in language specific blocks. Stimuli were intended to be
balanced per vocalic context.

The data were analyzed using generalized linear mixed effects models,
two sample tests of equivalence (Lakens, 2017) and two sample t-tests.
The results revealed that L3 relative VOT value fell inbetween L1 and L2
values, in line with previous research which found hybrid VOT values in
L3 learners. At the same time, practical equivalence was not found in
any language pair. However, additional post-hoc analyses of subsets of
the dataset revealed two distinct group trends. Out of 54 subset
participants, 18 showed evidence of primary L1 (Spanish) influence on L3
French productions, whereas the remaining 36 participants' L3 VOT was
more heavily affected by their L2. These two group trends were not,
however, equal in the magnitude of their effect. L1 influence was less
variable, where the L2 group showed simultaneous impact of both L1 and
L2, but a clear and heavy influence of the L2. The present data suggest
that, with an even higher sample size or meta-analysis, it is possible
that L3 learners choose just one primary source of influence on L3
productions at first exposure. However, the relative impact of that
primary choice of language is not equal in the case of the L1 and the
L2. The results of the present study underscore the need for higher
sample sizes in order to examine potential individual differences in
cross-linguistic influence of previously learned languages on initial L3
production.

\newpage

\textbf{Word List}

\begin{longtable}[]{@{}lll@{}}
\toprule
English & Spanish & French \\
\midrule
\endhead
tipping & tiro & tir \\
teller & tema & terre \\
tacky & talla & tasse \\
penny & quiso & quitte \\
pass & queja & quelle \\
parrot & cama & pile \\
kitten & piso & pere \\
kennel & pena & patte \\
cabbage & pato & \\
\bottomrule
\end{longtable}

\includegraphics[width=0.5\linewidth]{caspslap_2_files/figure-latex/figures-side-1}
\includegraphics[width=0.5\linewidth]{caspslap_2_files/figure-latex/figures-side-2}

\hypertarget{references}{%
\subsection{References}\label{references}}

\begingroup
\setlength{\parindent}{-0.5in}
\setlength{\leftskip}{0.5in}
\phantom{.}

\textcolor{white}{\\} \vspace{-0.5in}

\begin{tiny}
Brysbaert, M. (2020). Power considerations in bilingualism research: Time to step up our game. Bilingualism: Language and Cognition, 1–6. 

Lakens, D. (2017). Equivalence Tests: A Practical Primer for t Tests, Correlations, and Meta-Analyses. Social Psychological and Personality Science, 8(4), 355–362. 

Lemhfer, K., \& Broersma, M. (2012). Introducing LexTALE: A quick and valid Lexical Test for Advanced Learners of English. Behavior Research Methods, 44(2), 325–343. 

Llama, R., \& Cardoso, W. (2018). Revisiting (Non-)Native Influence in VOT Production: Insights from Advanced L3 Spanish. Languages, 3(3), 30. 

Llama, R., Cardoso, W., \& Collins, L. (2010). The influence of language distance and language status on the acquisition of L3 phonology. International Journal of Multilingualism, 7(1), 39–57. 

St\"{o}lten, K., Abrahamsson, N., \& Hyltenstam, K. (2015). EFFECTS OF AGE AND SPEAKING RATE ON VOICE ONSET TIME: The Production of Voiceless Stops by Near-Native L2 Speakers. Studies in Second Language Acquisition, 37(1), 71–100.

Wrembel, M. (2010). L2-accented speech in L3 production. International Journal of Multilingualism, 7(1), 75–90. 

Wrembel, M. (2011). Cross-linguistic Influence in Third Language Acquisition of Voice Onset Time. Proceedings of the 17th International Congress of Phonetic Sciences, 2157–2160.
\end{tiny}

\endgroup

\end{document}
